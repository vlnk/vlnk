\recipient{Pixmob}{103, Rue de Louvain\\ Montreal, QC, Canada, H2N 1A3}
\date{\today}
\opening{Madame, Monsieur,}
\closing{Dans l'attente de votre réponse pour un entretien, je vous prie d'agréer, Madame, Monsieur, mes sincères salutations.}
\enclosure[Ci-joint]{curriculum vit\ae{}}

\makelettertitle

\textbf{Poste : Développeur mobile (Android)}

\introduction{}
%%% PARTIE ENTREPRISE
Je postule à votre entreprise \textit{Pixmob}, car je suis très intéressé par votre proposition de \textbf{Développeur mobile}. Comme je l'indique dans mon CV je suis présentement en recherche de stage, mais mes compétences correspondent aux exigences du poste.
%%%

En effet, mes 3 années d'étude m'ont permis de travailler sur de nombreux projets d'envergures diverses dans de nombreux langages notamment en \textbf{C++, Python et C\#}. Mon cours de \textbf{Services Web} m'a permis de découvrir les possibilités offertes par le framework .NET grâce à des outils tels que \textit{Linq} et \textit{ASP.NET}. Par ailleurs, lors de mon stage de deuxième année j'ai travaillé sur un projet d'envergure en Allemagne durant 5 mois entièrement \textbf{en Anglais}. Ce projet consistait à \textbf{la réalisation d'une application de visualisation de données en C\# que nous avons entièrement codée du prototype à l'application finale pour le client.} Cette application devait être intégrée à une plateforme conçue par les ingénieurs d'\textit{Infineon Technologies AG} et je devais respecter leur guide de style pour développer mes composants avec \textit{WPF}. Ce projet m'a familiarisé avec l'utilisation des différents outils mis à disposition par \textit{Microsoft} et de la \textit{MSDN}.

De plus, durant ma deuxième année d'étude à Clermont-Ferrand puis au Canada, j'ai étudié le langage \textbf{Java} à tous les niveaux : nous avons étudié en profondeur les bases du langage puis nous avons étudié des outils plus avancés avec les cours de \textit{Java Avancé} et de \textit{Programation Objet Avancé}. Par exemple, nous avons étudié le \textbf{multithreading}, les \textbf{interfaces utilisateur}, la \textbf{réflexivité} ou encore les outils d'ingénieure tels que \textbf{Ant} ou \textbf{Jenkins}. À cela s'ajoute mon expérience en \textbf{C++} que j'ai renforcé en codant deux programmes pour la recherche opérationnelle en résolvant deux problèmes : le voyageur de commerce et le problème de couverture d'ensembles.

Pendant ma dernière année d'étude à Chicoutimi nous avons conçu une application Android. Cette application consiste en la recherche de correspondance entre des aliments scannés par leur codebar et les nombreuses recettes proposées sur le site \textit{Marmiton}. Nous avons conçu cette application à travers deux cours : un cours d'\textsc{uml} où nous avons écris tous les documents de conception (UseCase, Diagramme de Classes, Diagramme de Séquences) relatifs à l'application et un cours de Programmation Objet Avancé où nous avons programmé l'application. Tout le code de cette application est disponible sur mon \href{https://github.com/vlnk/ShootYourFridge}{\textit{GitHub}}. Ce développement m'a familiarisé avec la documentation pour Android, le développement d'interface GUI et de logique métier pour mobile. À cela s'ajoute, mes travaux effectués en cours d'\textit{Architecture d'Applications en Entreprises} qui m'ont permis d'étudier les solutions cross-plateformes tel que \textit{Ionic} afin de développer une application de navigation pour les aveugles. \conclusion{}

\makeletterclosing
