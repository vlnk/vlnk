\recipient{Groupe Trifide}{3700 Boulevard Wilfrid-Hamel\\
Québec, QC G1P 2J2\\
Canada}
\date{\today}
\opening{Madame, Monsieur,}
\closing{Dans l'attente de votre réponse pour un entretien, je vous prie d'agréer, Madame, Monsieur, mes sincères salutations.}
\enclosure[Ci-joint]{curriculum vit\ae{}}

\makelettertitle

\introduction{}
Double diplômant de l'école d'ingénieur française ISIMA et de l'université UQAC, je recherche un stage de 6 mois afin d'appliquer mes connaissances en entreprise. Je postule en candidature spontanée à votre entreprise \textit{Triffis Groupe}, car je suis très intéressé par vos tavaux en cartographie et du développement informatique qui se cache derrière cette technologie. Je me suis intéressé à la cartographie et aux outils logiciels associés grâce aux technologies développée par \textit{Mapzen} autour des cartes proposées par \textit{OpenStreetMap}. Ces technologies et le bénéfice qu'on peut avoir de ces outils sont indispensables pour s'orienter et optimiser ses trajets.

Présentement, mes 3 années d'étude m'ont permis de travailler sur de nombreux projets d'envergures diverses dans de nombreux langages notamment en \textbf{C++, Python et C\#}. Mon cours de \textbf{Services Web} m'a permis de découvrir les possibilités offertes par le framework .NET grâce à des outils tels que \textit{Linq} et \textit{ASP.NET}.  De même celui de \textbf{C\# avancé} m’a permis d'utiliser aisément toutes les fonctionnalités proposées par ce framework. J'ai exercé ces compétences à travers des exercices tels que la gestion de matchs de Quidditch dans un service web avec SOAP.

Par ailleurs, lors de mon stage de deuxième année j'ai travaillé sur un projet d'envergure en Allemagne durant 5 mois entièrement \textbf{en Anglais}. Ce projet consistait à \textbf{la réalisation d'une application de visualisation de données en C\# que nous avons entièrement codée du prototype à l'application finale pour le client.} Cette application devait être intégrée à une plateforme conçue par les ingénieurs d'\textit{Infineon Technologies AG} et je devais respecter leur guide de style pour développer mes composants avec \textit{WPF}. Ce projet m'a familiarisé avec l'utilisation des différents outils mis à disposition par \textit{Microsoft} et de la \textit{MSDN}.

J'ai assisté à un cours d'\textbf{Architecture des Applications en Entreprise} qui m'a familiarisé avec les concepts de SOA, d'architectures N-Tiers et de BPM par exemple. Ce cours a présenté l'ensemble des outils essentiels à la conception d'une architecture logicielle selon les critères de qualités nécessaires aux entreprises. Ainsi, j'ai étudié les méthodes d'analyse architecturale \textit{Architecture-Level Modifiability Analysis} et \textit{Architecture Trade-Off Analysis Method} dans le but de concevoir des architectures maintenables et qui correspondent aux exigences formulées par les clients. En outre, ce cours m'a permis d'approfondir mes connaissances dans les différents patrons de conception et surtout de les appliquer au sein d'architectures développées en \textit{Java}. Ceci associé à mon cours de \textbf{Programmation Objet Avancé}, j'ai été sensibilisé aux enjeux de réutilisation des différents composants au sein d'une entreprise et aux différents processus de tests et de revues de code. Ces compétences ont été acquises dans le but d'optimiser le développement des applications et de rendre ces applications plus maintenables et plus stables. Tous ces travaux m'ont amené à approfondir mes compétences en tant que programmeur C\# et analyste que je souhaite partager celles-ci avec votre équipe.
\conclusion{}
\makeletterclosing
