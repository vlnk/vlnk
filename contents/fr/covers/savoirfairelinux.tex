\recipient{Savoir Faire Linux}{7275 Saint Urbain, Bureau 200\\
Montréal, QC, H2R 2Y5\\
Canada}
\date{\today}
\opening{Madame, Monsieur,}
\closing{Dans l'attente de votre réponse pour un entretien, je vous prie d'agréer, Madame, Monsieur, mes sincères salutations.}
\enclosure[Ci-joint]{curriculum vit\ae{}}

\makelettertitle

\introduction{}
Je postule à votre offre de \textbf{Développeur Back End/Front End} car je suis très intéressé par votre travail de création d'outils \textit{Open Sources}. En effet, j'apprécie énormément cette philosophie et je la vois comme un moteur de l'innovation en informatique. J'utilise pour quasiment tous mes projets personnels \textit{GitHub} et je contribue à quelques projets car je pense que ce moyen de développement permet de développer des outils d'une importance capitale. La plupart des technologies actuelles intègrent des outils \textit{Open Sources} et c'est grâce à cette philosophie que chacun peut contribuer à l'amélioration de ces composants.

Présentement, mes 3 années d'étude m'ont permis de travailler sur de nombreux projets d'envergures diverses dans de nombreux langages notamment en \textbf{C++, Python et C\#}. Mon cours de \textbf{Services Web} m'a permis de découvrir les possibilités offertes par le framework .NET grâce à des outils tels que \textit{Linq} et \textit{ASP.NET}. Nous nous sommes exercés à utiliser ces outils en créant un service web de gestion de matchs et de joueurs de Quidditch. Par ailleurs, lors de mon stage de deuxième année j'ai travaillé sur un projet d'envergure en Allemagne durant 5 mois entièrement \textbf{en Anglais}. Ce projet consistait à \textbf{la réalisation d'une application de visualisation de données en C\# que nous avons entièrement codée du prototype à l'application finale pour le client.} Cette application devait être intégrée à une plateforme conçue par les ingénieurs d'\textit{Infineon Technologies AG} et je devais respecter leur guide de style pour développer mes composants avec \textit{WPF}. Ce projet m'a familiarisé avec l'utilisation des différents outils mis à disposition par \textit{Microsoft} et de la \textit{MSDN}.

J'ai travaillé en autodidacte sur mon site internet où j'ai exercé les langages JavaScript, HTML et CSS. Ce développement m'a permis de pratiquer mes connaissances dans ce langage et surtout d'améliorer mon statut de développeur web. En plus de ce travail, j'ai travaillé sur deux projets basés sur le langage \textbf{Python}. Le premier consistait à écrire un programme d'automatisation pour la recherche opérationnelle et analysait des programmes afin de paramétrer au mieux le logiciel \textsc{nomad} développé par des chercheurs de Montréal. En deuxième lieu, j'ai développé un programme de simulation de conversation \textsc{irc} pour un projet de web série intitulée \textit{Je suis toujours vivant}. Ce script utilise \textit{PyQt} pour facilité son utilisation. \conclusion{}

\makeletterclosing
