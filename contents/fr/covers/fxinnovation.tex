\recipient{FXinnovation}{400, boul. De Maisonneuve Ouest, bureau 1100\\
Montreal, Quebec, H3A 1L4\\
Canada}
\date{\today}
\opening{Madame, Monsieur,}
\closing{Dans l'attente de votre réponse pour un entretien, je vous prie d'agréer, Madame, Monsieur, mes sincères salutations.}
\enclosure[Ci-joint]{curriculum vit\ae{}}

\makelettertitle

\introduction{}
 Je postule en candidature spontanée à votre entreprise \textit{FXinnovation}, car je suis très intéressé par vos propositions d'\textbf{Analyste - Programmeur}. Les produits de \textit{Cloud-computing}, de \textit{Virtualisation} et de Monitoring que vous développez sont essentiels pour les entreprises qui souhaitent bénéficier de services personnalisés, performants et accessibles facilement via le \textit{Cloud}. Ces solutions adaptées aux exigences de vos clients leur permettent d'accroitre leur performance et leur contrôle sur leurs produits. Je suis très enthousiaste à l'idée de contribuer au développement de vos différents projets.

Présentement, mes 3 années d'étude m'ont permis de travailler sur de nombreux projets d'envergures diverses dans de nombreux langages notamment en \textbf{C++, Python et C\#}. Mes cours d'\textbf{Optimisation} et de \textbf{Métaheuristiques} m'ont sensibilisé à l'écriture d'algorithmes complexes en C++ afin de résoudre des problèmes théoriques en recherche opérationnelle. Ainsi, j'ai développé en deuxième année deux programmes permettant de résoudre certains problèmes de tournées de véhicules et ma troisième année d'étude m'a permis de développer un programme complet traitant le problème de couverture d'ensemble grâce à des articles de recherche préalablement choisis. Ces différents outils m'ont permis de me familiariser avec ce langage notamment avec la STL qui est un outil capital pour concevoir un code clair, concis et efficace. De plus, j'apprécie énormément les concepts liés à ce langage et je suis régulièrement les dernières avancées le concernant.

J'ai assisté à un cours d'\textbf{Architecture des Applications en Entreprise} qui m'a familiarisé avec les concepts de SOA, d'architectures N-Tiers et de BPM par exemple. Ce cours a présenté l'ensemble des outils essentiels à la conception d'une architecture logicielle selon les critères de qualités nécessaires aux entreprises. Ainsi, j'ai étudié les méthodes d'analyse architecturale \textit{Architecture-Level Modifiability Analysis} et \textit{Architecture Trade-Off Analysis Method} dans le but de concevoir des architectures maintenables et qui correspondent aux exigences formulées par les clients. En outre, ce cours m'a permis d'approfondir mes connaissances dans les différents patrons de conception et surtout de les appliquer au sein d'architectures développées en \textit{Java}. Ceci associé à mon cours de \textbf{Programmation Objet Avancé}, j'ai été sensibilisé aux enjeux de réutilisation des différents composants au sein d'une entreprise et aux différents processus de tests et de revues de code. Ces compétences ont été acquises dans le but d'optimiser le développement des applications et de rendre ces applications plus maintenables et plus stables. Nous y avons appris de nombreuses techniques avancées qui j'ai pu pratiquer sur mes comptes \href{https://github.com/vlnk/ShootYourFridge}{\textit{GitHub}} et \href{https://bitbucket.org/vlnk/tronpoa}{\textit{Bitbucket}}.

Par ailleurs, lors de mon stage de deuxième année j'ai travaillé sur un projet d'envergure en Allemagne durant 5 mois entièrement \textbf{en Anglais}. Ce projet consistait à \textbf{la réalisation d'une application de visualisation de données en C\# que nous avons entièrement codée du prototype à l'application finale pour le client.} Cette application devait être intégrée à une plateforme conçue par les ingénieurs d'\textit{Infineon Technologies AG} et je devais respecter leur guide de style pour développer mes composants avec \textit{WPF}. Ce projet m'a familiarisé avec l'élaboration d'interfaces graphiques et la séparation primordiale entre la partie \textit{design} et la partie \textit{logique} d'une application. J'ai complété cette expérience par la programmation d'un simulateur de conversation IRC. Ce programme a été codé en C++ et en Python avec l'utilisation de PyQt pour la GUI. \conclusion{}
\makeletterclosing
