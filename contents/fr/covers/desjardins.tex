\recipient{Mouvement des caisses Desjardins}{Complexe Desjardins, Rue Saint-Urbain \\
Montreal, Quebec, H5B 2B8\\
Canada}
\date{\today}
\opening{Madame, Monsieur,}
\closing{Dans l'attente de votre réponse pour un entretien, je vous prie d'agréer, Madame, Monsieur, mes sincères salutations.}
\enclosure[Ci-joint]{curriculum vit\ae{}}

\makelettertitle

Le Mouvement des caisses Desjardins est une entreprise très importante au Canada et je soutiens votre vision coopérative de la finance dans le but de créer une solidarité durable entre les individus et de les aider dans leur rapport à l'argent et à sa gestion. Je suis très enthousiaste à rejoindre cette vision et à y joindre mes compétences.

Je postule afin de rejoindre vos équipes dans le cadre de mon stage de fin d'étude d'ingénieur en Informatique. Je suis diplômant d'une Maîtrise en informatique à Chicoutimi et d'un diplôme d'ingénieur en France à l'\textit{Institut Supérieur d'Informatique, de Modélisation et de leurs Applications}. J'ai des compétences dans \textbf{le développement d'applications lourdes} en Java, C++ et C\#, dans l'\textbf{architecture des applications d'entreprise} avec la pratique d'UML, et de divers frameworks d'entreprise tels que Spring ou AspectJ. J'ai, de plus, une pratique dans les différentes méthodes de \textbf{gestion de projet} comme le SCRUM et XP ou de manière plus vaste, les processus AGILE. En plus de cet enseignement, j'ai pratiqué de nombreux outils Web avec l'utilisation de frameworks Ruby et JavaScript. Mon expérience dans ce domaine se reflète avec mon site personnel que je conçois cette année. Je développe ce site afin de partager mes travaux récents en Java, Android et C++ présents sur mon répertoire \textit{GitHub} et de parfaire mes connaissances dans ce domaine.

Par ailleurs, lors de mon stage de deuxième année j'ai travaillé sur un projet d'envergure en Allemagne durant 5 mois entièrement \textbf{en Anglais}. Ce projet consistait à \textbf{la réalisation d'une application de visualisation de données en C\# que nous avons entièrement codée du prototype à l'application finale pour le client.} Cette application devait être intégrée à une plateforme conçue par les ingénieurs d'\textit{Infineon Technologies AG} et je devais respecter leur guide de style pour développer mes composants avec \textit{WPF}. Ce projet m'a familiarisé avec l'élaboration d'interfaces graphiques et la séparation primordiale entre la partie \textit{design} et la partie \textit{logique} d'une application. J'ai complété cette expérience par la programmation d'un simulateur de conversation IRC. Ce programme a été codé en C++ et en Python avec l'utilisation de PyQt pour la GUI. \conclusion{}

\makeletterclosing
