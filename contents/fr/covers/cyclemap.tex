\recipient{Cycle Map}{}
\date{\today}
\opening{Madame, Monsieur,}
\closing{Dans l'attente de votre réponse pour un entretien, je vous prie d'agréer, Madame, Monsieur, mes sincères salutations.}
\enclosure[Ci-joint]{curriculum vit\ae{}}

\makelettertitle

\introduction{}
Je postule à votre offre de \textbf{Développeur Android} car je suis très intéressé par votre initiative de création d'application pour facilité le transport de chacun à vélo. En effet, j'ai moi-même pratiqué beaucoup de vélo notament en Allemagne où je me rendais chaque jour à mon travail avec mon vtt. J'avais réfléchi à une application mobile de rencontre à vélo après avoir rencontré de nombreux cyclistes en road trip à travers l'Allemagne dans l'auberge de jeunesse de Regensburg. Cette application est d'autant plus importante puisque les pistes cyclables sont très mal indiquées (notament en France), vous comblez ce problème avec votre application, l'omniprésence des smartphones et leur usage en constante augmentation.

Présentement, mes 3 années d'étude m'ont permis de travailler sur de nombreux projets d'envergures diverses dans de nombreux langages notamment en \textbf{C++, Python et C\#}. Mon cours de \textbf{Services Web} m'a permis de découvrir les possibilités offertes par le framework .NET grâce à des outils tels que \textit{Linq} et \textit{ASP.NET}. Nous nous sommes exercés à utiliser ces outils en créant un service web de gestion de matchs et de joueurs de Quidditch. Par ailleurs, lors de mon stage de deuxième année j'ai travaillé sur un projet d'envergure en Allemagne durant 5 mois entièrement \textbf{en Anglais}. Ce projet consistait à \textbf{la réalisation d'une application de visualisation de données en C\# que nous avons entièrement codée du prototype à l'application finale pour le client.} Cette application devait être intégrée à une plateforme conçue par les ingénieurs d'\textit{Infineon Technologies AG} et je devais respecter leur guide de style pour développer mes composants avec \textit{WPF}.

Pendant ma dernière année d'étude à Chicoutimi nous avons conçu une application Android. Cette application consiste en la recherche de correspondance entre des aliments scannés par leur codebar et les nombreuses recettes proposées sur le site \textit{Marmiton}. Nous avons conçu cette application à travers deux cours : un cours d'\textsc{uml} où nous avons écris tous les documents de conception (UseCase, Diagramme de Classes, Diagramme de Séquences) relatifs à l'application et un cours de Programmation Objet Avancé où nous avons programmé l'application. Tout le code de cette application est disponible sur mon \href{https://github.com/vlnk/ShootYourFridge}{\textit{GitHub}}. Ce développement m'a familiarisé avec la documentation pour Android, le développement d'interface GUI et de logique métier pour mobile. À cela s'ajoute, mes travaux effectués en cours d'\textit{Architecture d'Applications en Entreprises} qui m'ont permis d'étudier les solutions cross-plateformes tel que \textit{Ionic} afin de développer une application de navigation pour les aveugles. \conclusion{}

\makeletterclosing
