\recipient{Artika}{563 Avenue Lépine\\
Montréal, QC, H9P 2R2}
\date{\today}
\opening{Madame, Monsieur,}
\closing{Dans l'attente de votre réponse pour un entretien, je vous prie d'agréer, Madame, Monsieur, mes sincères salutations.}
\enclosure[Ci-joint]{curriculum vit\ae{}}

\makelettertitle

\textbf{Candidature spontannée : Développeur Android}

\introduction{}
%%% PARTIE ENTREPRISE
Je postule en candidature spontanée à votre entreprise \textbf{Artika} car je suis très intéressé par vos opportunités d'emplois pour \textbf{Developpeur Android} et \textbf{Développeur Web Back-End} au sein de votre nouvelle division \textit{SMARTIKA}. En effet, j'ai développé une application \textbf{Android} lors de ma dernière année d'étude : nous avons conçu tous les documents de conception relatifs à cette application lors d'un cours d'\textbf{UML} puis nous l'avons entièrement codée et présentée lors de notre cours de \textbf{Programmation Objet Avancée}. Cette application consiste en la recherche de recette de cuisine à partir des codes barres de produits.
%%%

Présentement, mes 3 années d'étude m'ont permis de travailler sur de nombreux projets d'envergures diverses dans de nombreux langages notamment en \textbf{C++, Python et C\#}. Mon cours de \textbf{Services Web} m'a permis de découvrir les possibilités offertes par le framework .NET grâce à des outils tels que \textit{Linq} et \textit{ASP.NET}. Par ailleurs, lors de mon stage de deuxième année j'ai travaillé sur un projet d'envergure en Allemagne durant 5 mois entièrement \textbf{en Anglais}. Ce projet consistait à \textbf{la réalisation d'une application de visualisation de données en C\# que nous avons entièrement codée du prototype à l'application finale pour le client.} Cette application devait être intégrée à une plateforme conçue par les ingénieurs d'\textit{Infineon Technologies AG} et je devais respecter leur guide de style pour développer mes composants avec \textit{WPF}. Ce projet m'a familiarisé avec l'utilisation des différents outils mis à disposition par \textit{Microsoft} et de la \textit{MSDN}.

J'ai travaillé en autodidacte sur mon site internet où j'ai exercé les langages JavaScript, HTML et CSS. Ce développement m'a permis de pratiquer mes connaissances dans ce langage et surtout d'améliorer mon statut de développeur web. En plus de ce travail, j'ai travaillé sur deux projets basés sur le langage \textbf{Python}. Le premier consistait à écrire un programme d'automatisation pour la recherche opérationnelle et analysait des programmes afin de paramétrer au mieux le logiciel \textsc{nomad} développé par des chercheurs de Montréal. En deuxième lieu, j'ai développé un programme de simulation de conversation \textsc{irc} pour un projet de web série intitulée \textit{Je suis toujours vivant}. Ce script utilise \textit{PyQt} pour faciliter son utilisation. Enfin, je mène une veille informatique pour ma culture personnelle ce qui m'a permis de découvrir de nombreux outils innovants en informatique notamment via la \textit{Hacker News}. Cette veille m'a permis d'utiliser le langage \textbf{Ruby} dans la conception de mon site Internet et a nettement accéléré ma gestion des bases de données.

\conclusion{}

\makeletterclosing
